\documentclass[12pt]{amsart}
\usepackage{geometry}                % See geometry.pdf to learn the layout options. There are lots.
\geometry{a4paper, left=25mm, right=30mm}                   % ... or a4paper or a5paper or ... 
%\geometry{landscape}                % Activate for for rotated page geometry
%\usepackage[parfill]{parskip}    % Activate to begin paragraphs with an empty line rather than an indent
\usepackage[ngerman]{babel}
\usepackage{multicol}
\usepackage{graphicx}
\usepackage{amssymb}
\usepackage{epstopdf}
\usepackage{relsize}
\usepackage{color}
%\usepackage[framed,numbered,autolinebreaks,useliterate]{mcode}
\usepackage[normalem]{ulem}



% f�r Code:
\usepackage{listings}
\usepackage{color}
\definecolor{keywcolor}{rgb}{0.6,0,0.5}
\definecolor{dkgreen}{rgb}{0,1,0}
\lstset{ %
  linewidth=\textwidth,
  language=C,                % the language of the code
  basicstyle=\footnotesize,           % the size of the fonts that are used for the code
  numbers=left,                   % where to put the line-numbers
  numberstyle=\tiny\color{black},  % the style that is used for the line-numbers
  stepnumber=1,                   % the step between two line-numbers. If it's 1, each line 
                                 		 % will be numbered
  numbersep=5pt,                  % how far the line-numbers are from the code
  backgroundcolor=\color{white},      % choose the background color. You must add \usepackage{color}
  showspaces=false,               % show spaces adding particular underscores
  showstringspaces=false,         % underline spaces within strings
  showtabs=false,                 % show tabs within strings adding particular underscores
  frame=single,                   % adds a frame around the code
  rulecolor=\color{black},        % if not set, the frame-color may be changed on line-breaks within not-black text (e.g. comments (green here))
  tabsize=4,                      % sets default tabsize to 2 spaces
  captionpos=b,                   % sets the caption-position to bottom
  breaklines=true,                % sets automatic line breaking
  breakatwhitespace=false,        % sets if automatic breaks should only happen at whitespace
  keywordstyle=\color{keywcolor},          % keyword style
  commentstyle=\color{blue},       % comment style
}



\title{bwview Formeln}
\author{Martin Hansen}
%\date{11. Oktober 2012}                                          % Activate to display a given date or no date


\begin{document}
%\maketitle

\renewcommand{\baselinestretch}{1.50}\normalsize

\section{"Uber dieses Dokument}

Dieses Dokument ist \emph{work in progress} und dient vorerst nur der Sammlung diverser (aus dem Quellcode von bwview extrahierter) Formeln.

\section{Nutzer-Parameter}

Eine (\emph{noch unvollst"andige}) Sammlung der Parameter, die der Nutzer manipulieren kann:

\subsection{Gain for signal display (Taste s)} Wirkt sich nur auf die (untransformierte) Zeit-Amplitude-Darstellung am oberen Rand des Fensters, nicht auf die Signalanalyse aus. Wert ist gespeichert in der globalen Variable \emph{s\_gain}. Verwendet wird dieser Wert in der Funktion \emph{void draw\_signal(BWAnal *aa)} in \emph{display.c}.

\subsection{Brightness of main display (Taste b)} Wirkt sich auf die Darstellung im Hauptfenster (nicht auf die Signalanalyse) aus.  Wert ist gespeichert in der globalen Variable \emph{s\_bri}. Verwendet wird dieser Wert in der Funktion \emph{void draw\_mag\_lines(BWAnal *aa, int lin, int cnt)} in \emph{display.c} - dort werden die Ergebniswerte der Analyse vor dem Plotten durch Multiplikation mit  \emph{s\_bri} skaliert .

\section{analysis.c}

Die Datei \emph{analysis.c} enth"alt die Funktionen zur Analyse der EEG-Daten.


\subsection{Vorbemerkung: Analysis types} Das Tool bietet drei verschiedene Analyse-Typen:

\begin{itemize}
\item 0 : Blackman-Fenster (default)
\item 1 : IIR Biquad Filter mit $Q=0,5$
\item 2 : IIR Biquad Filter mit $Q=0,72$
\end{itemize} 
Im folgenden wird vorerst nur die Analyse mit Analyse-Typ 0 (Blackman-Fenster) betrachtet, da wir in der Anwendung bisher auch nur diese genutzt haben.

\subsection{struct BWSetup} Ein Struct mit den grundlegenden Ausf"uhrungsparametern:
\subsubsection{Definition} Der Struct ist folgenderma"sen definiert:

\begin{lstlisting}[firstnumber=91]
struct BWSetup {
    int typ;         // Analyse-Typ (0-2, s.o.)
    int off;         // Offset im input file (in samples, gezaehlt ab 0)
    int chan;        // anzuzeigender Kanal (gezaehlt ab 0)
    int tbase;       // Time-base (Samples pro Datenpunkt (horizontal))
    int sx;          // Zahl der zu berechnenden Spalten (size-X)
    int sy;          // Zahl der zu berechnenden Zeilen (size-Y)
    double freq0, freq1; // oberste und unterste Frequenz (Achtung: s.u.)
    double wwrat;    // Verhaeltnis von Fensterbreite zur Wellenlaenge der Mittenfrequenz
};
\end{lstlisting}

\subsubsection{Zeilen- und Spaltenzahl} Die Werte f"ur \emph{sx} und \emph{sy} entsprechen der Anzahl aktuell darstellbarer Pixel (verstellbar durch Skalierung des GUI-Fensters).

\subsubsection{Oberste und unterste Frequenz}Auf die Zeilen (Anzahl: \emph{sy}) werden "aqudistante Frequenzb"ander (\emph{noch zu "ubersetzen: in log-freq-space}) zwischen \emph{freq0} und \emph{freq1} verteilt. Das bedeutet, dass das oberste Band knapp unter \emph{freq0} und das unterste Band knapp "uber \emph{freq1} liegt.
Um also z.B. sechs B"ander zwischen 128 Hz und 256 Hz zu erhalten, setzt man$BWSetup.sy=6$, $BWSetup.freq0=256$, $BWSetup.freq1=128$.


\subsection{struct BWAnal} Ein Struct mit detaillierten Parametern f"ur die Analyse sowie den entsprechenden Daten-Arrays:
 
\subsubsection{Definition} Der Struct ist folgenderma"sen definiert:

\begin{lstlisting}[firstnumber=109]
struct BWAnal {
    BWFile *file;
    BWBlock **blk;   // Liste der geladenen Bloecke
    int n_blk;       // Anzahl Bloecke in der Liste 
    int bsiz;        // Blockgroesse
    int bnum;        // Nummer des vordersten Blocks in der Liste
    int half;        // Benoetigt dieser Filter nur die linke Haelfte der Daten ? 0 Nein, 1 Ja

    fftw_plan *plan; // Big list of FFTW plans (see note below for ordering)
    int m_plan;      // Maximum plans (i.e. allocated size of plan[])

    int inp_siz;     // Size of data in inp[], or 0 if not valid
    fftw_real *inp;  // FFT'd input data (half-complex)
    fftw_real *wav;  // FFT'd wavelet (real)
    fftw_real *tmp;  // General workspace (complex), also used by IIR
    fftw_real *out;  // Output (complex)

    // "Oeffentlich" lesbare, nicht veraenderliche Informationen:
    int n_chan;      // Anzahl Kanaele in Input-Datei
    double rate;     // Abtastrate

    // "Oeffentlich" lesbare, sich veraendernde Informationen:
    BWSetup c;       // Aktuelles Setup
    float *sig;      // Signal mid-point values: sig[x], or NAN for sync errors
    float *sig0;     // Signal minimum values: sig0[x], or NAN for sync errors
    float *sig1;     // Signal maximum values: sig1[x], or NAN for sync errors
    float *mag;      // Magnitude information: mag[x+y*sx] - dies ist der Wert, der am Ende geplottet wird
    float *est;      // Estimated nearby peak frequencies: est[x+y*sx], or NAN if can't calc
    float *freq;     // Centre-frequency of each line (Hz): freq[y]
    float *wwid;     // Logical width of window in samples: wwid[y]
    int *awwid;      // Actual width of window, taking account of IIR tail: awwid[y]
    int *fftp;       // FFT plan to use (index into ->plan[], fftp[y]%3==0)
    double *iir;     // IIR filter coefficients: iir[y*3], iir[y*3+1], iir[y*3+2]
    int yy;      // Number of lines currently correctly calculated in arrays
    int sig_wind;    // Are the ->sig arrays windowed ? 0 no, 1 yes
   
    // "Oeffentlich" schreibbare Informationen:
    BWSetup req;     // Angefordertes Setup
};
\end{lstlisting}

\subsection{Makro PLAN\_SIZE(n)} Ein Makro zur Berechnung, wie gro"s dass Array f"ur einen fftw-\emph{plan} sein muss. 

$n \mod 3$ bestimmt den Typen des Plans: 

\begin{itemize}
\item 0 : real $\mapsto$ komplex
\item 1 : komplex $\mapsto$ real
\item 2 : komplex $\mapsto$ komplex
\end{itemize} 

Das Makro berechnet:

$PLAN\_SIZE(n) = \begin{cases}
  3 \cdot 2^{\lfloor \frac{n}{6} \rfloor },  & \text{wenn }n\mod 3 \text{ ungerade.}\\
  2 \cdot 2^{\lfloor \frac{n}{6} \rfloor },  & \text{wenn }n\mod 3 \text{ gerade.}
\end{cases}$


\subsection{static void copy\_samples(BWAnal *aa, fftw\_real *arr, int off, int chan, int len, int errors)} Eine Funktion, die den mit \emph{off} und\emph{len} ausgew"ahlten Bereich der EEG-Daten eines Kanals (\emph{chan}) einem \emph{BWBlock (struct definiert in file.c, Details sind noch to do)} in ein Array kopiert. Geht der gew"ahlte Bereich "uber die Grenzen der Datei hinaus, werden entsprechend Nullen in das Array eingetragen.

\emph{Anmerkung: Code ist nicht ins Detail "uberpr"uft, die Ausf"uhrungen beruhen nur auf den Kommentaren und einem groben Blick auf den Code.}


\subsection{static void recreate\_arrays(BWAnal *aa)} F"uhrt erst einmal \emph{free} auf alle Arrays des BWAnal-Structs aus, um sie dann anhand der im Struct eingetragenen Zeilen- und Spaltengr"o"sen (\emph{aa-$>$c.sx und aa-$>$c.sy}) neu zu allokieren.


\subsection{BWAnal * bwanal\_new(char *fmt, char *fnam)} Erzeugt einen neuen BWAnal-Struct f"ur die Datei \emph{fnam}, die mit den in \emph{fmt} "ubergebenen Formatspezifikationen geladen wird. Dabei wird (neben einigen Default-Parametern, die sp"ater wieder "uberschrieben werden) folgender Parameter gesetzt:

Block size $bsiz=1024$


\subsection{void bwanal\_start(BWAnal *aa)} Die Funktion \emph{bwanal\_start(BWAnal *aa)} startet die Berechnungen. Sie wird aufgerufen von der \emph{main}-Funktion (siehe \emph{bwview.c}).

\emph{Vorbemerkung:} Von den drei

$sx:= \text{Anzahl Spalten}$

$sy:= \text{Anzahl Zeilen}$

$tbase:= \text{Samples pro Datenpunkt/Pixel}$

$wwrat:= \text{Ratio of window width to the centre-frequency wavelength}$

$log0 = \ln(f_{MAX})$

$log1 = \ln(f_{MIN})$

\textbf{F"ur jede Zeile $a<sy$:\{}

Mittenfrequenz: $\displaystyle{f_a= e^{log0 + \frac{a+0,5}{sy} \left(log1-log0\right)}}$

(Logische) Fenstergr"o"se in Samples: $wwid_a= \frac{f_{SAMPLE}}{f_a} wwrat $

$siz= sx\cdot tbase + \lfloor wwid_a \rfloor + 2 + 10$

$b= \lfloor \log_2\left(siz\right)\rfloor \cdot 6$

Solange  $PLAN\_SIZE(b) > siz$: \{ $b=b-1$\}

Index auf zu benutzenden fft Plan: $fftp_a = b$

Dazugeh"orige Fenstergr"o"se: $awwid_a=PLAN\_SIZE(b)$

Maximale Plangr"o"se: $maxsiz= PLAN\_SIZE(b)$;

\textbf{\} // Ende f"ur jede Zeile}

$a = fftp_{sy-1} + 3$

falls $a > mplan$: allokiere die Plan-Liste $plan$ neu f"ur $a$ Pl"ane, kopiere die Pl"ane aus der alten Liste hinein (kurzum: vergr"ossere $plan$, um Platz f"ur $a$ Pl"ane zu bieten).

\textbf{F"ur jede Zeile $a<sy$:\{}

$ii = fftp_a$

$siz = PLAN\_SIZE(ii)$

\textbf{F"ur jedes Zeile $b\in\left[0,1,2\right]$:\{}

$c =  ii + b$

\textbf{wenn plan\_{c} noch nicht existiert \{}

wenn $b=0$ : $plan_c = $\emph{rfftw\_create\_plan(siz, FFTW\_REAL\_TO\_COMPLEX, FFTW\_ESTIMATE $\vert$ FFTW\_USE\_WISDOM)}

wenn $b=1$ : $plan_c = $\emph{rfftw\_create\_plan(siz, FFTW\_COMPLEX\_TO\_REAL, FFTW\_ESTIMATE $\vert$ FFTW\_USE\_WISDOM)}

wenn $b=2$ : $plan_c = $\emph{fftw\_create\_plan(siz, FFTW\_BACKWARD, FFTW\_ESTIMATE $\vert$ FFTW\_USE\_WISDOM)}

\textbf{\}} // Ende wenn

\textbf{\} // Ende f"ur jedes $b$}

\textbf{\} // Ende f"ur jede Zeile}

\emph{to be continued...}



\end{document}
