\documentclass[12pt]{amsart}
\usepackage{geometry}                % See geometry.pdf to learn the layout options. There are lots.
\geometry{a4paper, left=25mm, right=30mm}                   % ... or a4paper or a5paper or ... 
%\geometry{landscape}                % Activate for for rotated page geometry
%\usepackage[parfill]{parskip}    % Activate to begin paragraphs with an empty line rather than an indent
\usepackage[ngerman]{babel}
\usepackage{multicol}
\usepackage{graphicx}
\usepackage{amssymb}
\usepackage{epstopdf}
\usepackage{relsize}
\usepackage{color}
%\usepackage[framed,numbered,autolinebreaks,useliterate]{mcode}
\usepackage[normalem]{ulem}

\title{bwview Formeln}
\author{Martin Hansen}
%\date{11. Oktober 2012}                                          % Activate to display a given date or no date


\begin{document}
%\maketitle

\renewcommand{\baselinestretch}{1.50}\normalsize

\section{"Uber dieses Dokument}

Dieses Dokument ist \emph{work in progress} und dient vorerst nur der Sammlung diverser (aus dem Quellcode von bwview extrahierter) Formeln.

\section{analysis.c}

Die Datei \emph{analysis.c} enth"alt die Funktionen zur Analyse der EEG-Daten.


\subsection{Vorbemerkung Analysis types} Das Tool bietet drei verschiedene Analyse-Typen:

\begin{itemize}
\item 0 : Blackman-Fenster (default)
\item 1 : IIR Biquad Filter mit $Q=0,5$
\item 2 : IIR Biquad Filter mit $Q=0,72$
\end{itemize} 
Im folgenden wird vorerst nur die Analyse mit Analyse-Typ 0 (Blackman-Fenster) betrachtet, da wir in der Anwendung bisher auch nur diese genutzt haben.


\subsection{Makro PLAN\_SIZE(n)} Ein Makro zur Berechnung, wie gro"s dass Array f"ur einen fftw-\emph{plan} sein muss. 

$n \mod 3$ bestimmt den Typen des Plans: 

\begin{itemize}
\item 0 : real $\mapsto$ komplex
\item 1 : komplex $\mapsto$ real
\item 2 : komplex $\mapsto$ komplex
\end{itemize} 

Das Makro berechnet:

$PLAN\_SIZE(n) = \begin{cases}
  3 \cdot 2^{\lfloor \frac{n}{6} \rfloor },  & \text{wenn }n\mod 3 \text{ ungerade.}\\
  2 \cdot 2^{\lfloor \frac{n}{6} \rfloor },  & \text{wenn }n\mod 3 \text{ gerade.}
\end{cases}$


\subsection{static void copy\_samples(BWAnal *aa, fftw\_real *arr, int off, int chan, int len, int errors)} Eine Funktion, die den mit \emph{off} und\emph{len} ausgew"ahlten Bereich der EEG-Daten eines Kanals (\emph{chan}) aus der Eingabe-Datei in ein Array kopiert. Geht der gew"ahlte Bereich "uber die Grenzen der Datei hinaus, werden entsprechend Nullen in das Array eingetragen.

\emph{Anmerkung: Code ist nicht ins Detail "uberpr"uft, die Ausf"uhrungen beruhen nur auf den Kommentaren und einem groben Blick auf den Code (was sollte da auch sonst passieren...).}


\subsection{void bwanal\_start(BWAnal *aa)} Die Funktion \emph{bwanal\_start(BWAnal *aa)} Startet die Berechnungen.

\emph{Vorbemerkung:} Von den drei

$sx:= \text{Anzahl Spalten}$

$sy:= \text{Anzahl Zeilen}$

$tbase:= \text{Samples pro Datenpunkt/Pixel}$

$wwrat:= \text{Ratio of window width to the centre-frequency wavelength}$

$log0 = \ln(f_{MAX})$

$log1 = \ln(f_{MIN})$

\textbf{F"ur jede Zeile $a<sy$:\{}

Mittlere Frequenz (?) der Zeile: $\displaystyle{f_a= e^{log0 + \frac{a+0,5}{sy} \left(log1-log0\right)}}$

(Logische) Fenstergr"o"se in Samples: $wwid_a= \frac{f_{SAMPLE}}{f_a} wwrat $

$siz= sx\cdot tbase + \lfloor wwid_a \rfloor + 2 + 10$

$b= \lfloor \log_2\left(siz\right)\rfloor \cdot 6$

\textbf{Solange  $PLAN\_SIZE(b) > siz$: \{} $b=b-1$\textbf{\}}

Index auf zu benutzenden fft Plan: $fftp_a = b$

Dazugeh"orige Fenstergr"o"se: $awwid_a=PLAN\_SIZE(b)$

Maximale PLangr"o"se: $maxsiz= PLAN\_SIZE(b)$;

\textbf{\}}

\emph{to be continued...}


\subsection{static void recreate\_arrays(BWAnal *aa)} F"uhrt erst einmal \emph{free} auf alle Arrays des BWAnal-Structs aus, um sie dann anhand der im Struct eingetragenen Zeilen- und Spaltengr"o"sen (\emph{aa->c.sx und aa->c.sy}) neu zu allokieren.
\end{document}
