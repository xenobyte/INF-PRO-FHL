\documentclass[12pt]{amsart}
\usepackage{geometry}                % See geometry.pdf to learn the layout options. There are lots.
\geometry{a4paper, left=25mm, right=30mm}                   % ... or a4paper or a5paper or ... 
%\geometry{landscape}                % Activate for for rotated page geometry
%\usepackage[parfill]{parskip}    % Activate to begin paragraphs with an empty line rather than an indent
\usepackage[ngerman]{babel}
\usepackage{multicol}
\usepackage{graphicx}
\usepackage{amssymb}
\usepackage{epstopdf}
\usepackage{relsize}
\usepackage{color}
%\usepackage[framed,numbered,autolinebreaks,useliterate]{mcode}
\usepackage[normalem]{ulem}



% f�r Code:
\usepackage{listings}
\usepackage{color}
\definecolor{keywcolor}{rgb}{0.6,0,0.5}
\definecolor{dkgreen}{rgb}{0,1,0}
\lstset{ %
  linewidth=15.7cm,
  language=C,                % the language of the code
  basicstyle=\footnotesize,           % the size of the fonts that are used for the code
  numbers=left,                   % where to put the line-numbers
  numberstyle=\tiny\color{black},  % the style that is used for the line-numbers
  stepnumber=1,                   % the step between two line-numbers. If it's 1, each line 
                                 		 % will be numbered
  numbersep=5pt,                  % how far the line-numbers are from the code
  backgroundcolor=\color{white},      % choose the background color. You must add \usepackage{color}
  showspaces=false,               % show spaces adding particular underscores
  showstringspaces=false,         % underline spaces within strings
  showtabs=false,                 % show tabs within strings adding particular underscores
  frame=single,                   % adds a frame around the code
  rulecolor=\color{black},        % if not set, the frame-color may be changed on line-breaks within not-black text (e.g. comments (green here))
  tabsize=4,                      % sets default tabsize to 2 spaces
  captionpos=b,                   % sets the caption-position to bottom
  breaklines=true,                % sets automatic line breaking
  breakatwhitespace=false,        % sets if automatic breaks should only happen at whitespace
  keywordstyle=\color{keywcolor},          % keyword style
  commentstyle=\color{blue},       % comment style
}



\title{bwview Formeln}
\author{Martin Hansen}
%\date{11. Oktober 2012}                                          % Activate to display a given date or no date


\begin{document}
%\maketitle

\renewcommand{\baselinestretch}{1.50}\normalsize

\section{"Uber dieses Dokument}

Dieses Dokument ist \emph{work in progress} und dient vorerst nur der Sammlung diverser (aus dem Quellcode von bwview extrahierter) Formeln.

\section{analysis.c}

Die Datei \emph{analysis.c} enth"alt die Funktionen zur Analyse der EEG-Daten.


\subsection{Vorbemerkung: Analysis types} Das Tool bietet drei verschiedene Analyse-Typen:

\begin{itemize}
\item 0 : Blackman-Fenster (default)
\item 1 : IIR Biquad Filter mit $Q=0,5$
\item 2 : IIR Biquad Filter mit $Q=0,72$
\end{itemize} 
Im folgenden wird vorerst nur die Analyse mit Analyse-Typ 0 (Blackman-Fenster) betrachtet, da wir in der Anwendung bisher auch nur diese genutzt haben.

\subsection{struct BWSetup} Ein Struct mit den grundlegenden Parametern:

\begin{lstlisting}
typedef struct BWSetup BWSetup;

struct BWSetup {
    int typ;         // Analyse-Typ (0-2, s.o.)
    int off;         // Offset im input file (in samples, gezaehlt ab 0)
    int chan;        // anzuzeigender Kanal (gezaehlt ab 0)
    int tbase;       // Time-base (Samples pro Datenpunkt (horizontal))
    int sx;          // Zahl der zu berechnenden Spalten (size-X)
    int sy;          // Zahl der zu berechnenden Spalten (size-Y)
    double freq0, freq1; // oberste und unterste Frequenz (Achtung: s.u.)
    double wwrat;    // Verhaeltnis von Fensterbreite zur Wellenlaenge der Mittenfrequenz
};

\end{lstlisting}

\subsection{Oberste und unterste Frequenz} Erl"auterung zu den beiden Frequenzen \emph{freq0} und \emph{freq1} aus dem struct BWSetup (s.o.):
Auf die Zeilen (Anzahl: BWSetup.sy) werden "aqudistante Frequenzb"ander (\emph{noch zu "ubersetzen: in log-freq-space}) zwischen \emph{freq0} und \emph{freq1} verteilt. Das bedeutet, dass das oberste Band knapp unter \emph{freq0} und das unterste Band knapp "uber \emph{freq1} liegt.
Um also z.B. sechs B"ander zwischen 128 Hz und 256 Hz zu erhalten, setzt man$BWSetup.sy=6$, $BWSetup.freq0=128$, $BWSetup.freq1=256$.

\emph{\textbf{Achtung(!):} obiges ist eine "Ubersetzung der Kommentare, die etwas widerspr"uchlich sind, was \emph{freq0} und \emph{freq1} angeht (und welches von beiden die obere und welches die untere Frequenz ist). \textbf{Nach allem, was bisher vom Code gesichtet wurde, ist  \emph{freq0} die untere und \emph{freq1} die obere Frequenz} (das Bespiel stimmt also, aber der Satz davor nicht).}


\subsection{Makro PLAN\_SIZE(n)} Ein Makro zur Berechnung, wie gro"s dass Array f"ur einen fftw-\emph{plan} sein muss. 

$n \mod 3$ bestimmt den Typen des Plans: 

\begin{itemize}
\item 0 : real $\mapsto$ komplex
\item 1 : komplex $\mapsto$ real
\item 2 : komplex $\mapsto$ komplex
\end{itemize} 

Das Makro berechnet:

$PLAN\_SIZE(n) = \begin{cases}
  3 \cdot 2^{\lfloor \frac{n}{6} \rfloor },  & \text{wenn }n\mod 3 \text{ ungerade.}\\
  2 \cdot 2^{\lfloor \frac{n}{6} \rfloor },  & \text{wenn }n\mod 3 \text{ gerade.}
\end{cases}$


\subsection{static void copy\_samples(BWAnal *aa, fftw\_real *arr, int off, int chan, int len, int errors)} Eine Funktion, die den mit \emph{off} und\emph{len} ausgew"ahlten Bereich der EEG-Daten eines Kanals (\emph{chan}) einem \emph{BWBlock (struct definiert in file.c, Details sind noch to do)} in ein Array kopiert. Geht der gew"ahlte Bereich "uber die Grenzen der Datei hinaus, werden entsprechend Nullen in das Array eingetragen.

\emph{Anmerkung: Code ist nicht ins Detail "uberpr"uft, die Ausf"uhrungen beruhen nur auf den Kommentaren und einem groben Blick auf den Code.}


\subsection{static void recreate\_arrays(BWAnal *aa)} F"uhrt erst einmal \emph{free} auf alle Arrays des BWAnal-Structs aus, um sie dann anhand der im Struct eingetragenen Zeilen- und Spaltengr"o"sen (\emph{aa-$>$c.sx und aa-$>$c.sy}) neu zu allokieren.


\subsection{BWAnal * bwanal\_new(char *fmt, char *fnam)} Erzeugt einen neuen BWAnal-Struct f"ur die Datei \emph{fnam}, die mit den in \emph{fmt} "ubergebenen Formatspezifikationen geladen wird. Dabei wird (neben einigen Default-Parametern, die sp"ater wieder "uberschrieben werden) folgender Parameter gesetzt:

Block size $bsiz=1024$


\subsection{void bwanal\_start(BWAnal *aa)} Die Funktion \emph{bwanal\_start(BWAnal *aa)} Startet die Berechnungen.

\emph{Vorbemerkung:} Von den drei

$sx:= \text{Anzahl Spalten}$

$sy:= \text{Anzahl Zeilen}$

$tbase:= \text{Samples pro Datenpunkt/Pixel}$

$wwrat:= \text{Ratio of window width to the centre-frequency wavelength}$

$log0 = \ln(f_{MAX})$

$log1 = \ln(f_{MIN})$

\textbf{F"ur jede Zeile $a<sy$:\{}

Mittenfrequenz: $\displaystyle{f_a= e^{log0 + \frac{a+0,5}{sy} \left(log1-log0\right)}}$

(Logische) Fenstergr"o"se in Samples: $wwid_a= \frac{f_{SAMPLE}}{f_a} wwrat $

$siz= sx\cdot tbase + \lfloor wwid_a \rfloor + 2 + 10$

$b= \lfloor \log_2\left(siz\right)\rfloor \cdot 6$

\textbf{Solange  $PLAN\_SIZE(b) > siz$: \{} $b=b-1$\textbf{\}}

Index auf zu benutzenden fft Plan: $fftp_a = b$

Dazugeh"orige Fenstergr"o"se: $awwid_a=PLAN\_SIZE(b)$

Maximale Plangr"o"se: $maxsiz= PLAN\_SIZE(b)$;

\textbf{\}}

\emph{to be continued...}



\end{document}
