\documentclass[german,ignorenonframetext]{beamer}

\usepackage[ngerman]{babel} % f�r deutsche Spracheinstellungen
\usepackage{graphics} % um Bilder einbinden zu k�nnen
\usepackage[dvips]{epsfig} % um Bilder zu skalieren
\usepackage[applemac]{inputenc} % inputencoding (MacOS) f�r Umlaute und Akzente

%%%%%%%%%%%%%%%%%%%%%%%%%%%%%%%%%%%%%%%%%%%

%%%%%%% Die folgenden Befehle definieren das Grundlayout, blenden auf der
%%%%%%% Titelseite die HAW-Infos ein und setzen das 
%%%%%%% HAW-Logo in die Ecke
\mode<presentation>{\usetheme{Berkeley}}
\logo{\pgfimage[height=1.5cm]{HAW_wuerfel+}}
\institute[MT -- HAW Hamburg]{HAW Hamburg\\ Fakult�t DMI, Dept.\ Medientechnik}

%%%%%%% der folgende Befehl l�sst die mit "\pause" verdeckten Teile der Folien 
%%%%%%% transparent erscheinen.  
\setbeamercovered{transparent}  

%%%%%%% die folgende Sequenz blendet mit jeder neuen Section einmal das 
%%%%%%% Inhaltsverzeichnis mit dem Titel "�bersicht" ein und markiert den
%%%%%%% jeweils aktuellen Gliederungspunkt
\AtBeginSection[]{
\begin{frame}<beamer>
\frametitle{�bersicht} 
\tableofcontents[currentsection,currentsubsection]
\end{frame}
}

%%%%%%%%%%%%%%%%%%%%%%%%%%%%%%%%%%%%%%%%%%%

%%%%%%% jeder dieser Titelseiten-Befehle kennt eine in eckige Klammern gesetzte 
%%%%%%% Kurzform, die im Rand benutzt wird
\title[�berschrift]{Die Haupt�berschrift}
\subtitle{Und eine Unter--�berschrift}
\author[Name]{Vorname Nachname}
\date{\today}

%%%%%%%%%%%%%%%%%%%%%%%%%%%%%%%%%%%%%%%%%%%

\begin{document}

%%%%%% dieser Befehl erzeugt das Deckblatt. 
%%%%%% Mit der Option plain wird das Layout f�r das Deckblatt abgeschaltet
%\frame[plain]{\titlepage}
\frame{\titlepage}

\begin{frame}
  \frametitle{�bersicht}
  \tableofcontents
\end{frame}

%%%%%%%%%%%%%%%%%%%%%%%%%%%%%%%%%%%%%%%%%%%
\section{Erster Abschnitt} % die sections sind hier nur zum Strukturieren des Vortrags!!
%% der Name der section taucht nur im Inhaltsverzeichnis und im linken Rand auf.
%%%%%%%%%%%%%%%%%%%%%%%%%%%%%%%%%%%%%%%%%%%
\begin{frame}
\frametitle{Erste Folie im ersten Abschnitt}
Der sukzessive Aufbau einer Folie aus Teil--Folien geht am einfachsten mit dem pause--Befehl.

\pause
\begin{itemize}
\item Bla blabla
    \begin{itemize} 
    \item und gleich mal eine verschachtelte
    \item Aufz�hlung
    \end{itemize}
\pause
\item Blabla bla blabla
\pause
\item Blabla
\end{itemize}
\end{frame}


\begin{frame}
\frametitle{Zweite Folie im ersten Abschnitt}
\pause 
\begin{itemize}
\item \pause \textbf{Ha\pause nu\pause ta} \pause(Haselnusstafel):\\ \pause
    {\em der pause-Befehl kann �berall gesetzt werden!}
\pause
\item \textbf{UKW} (Ultrakurzwelle):\\
    {\em Hier kommt noch Text rein}
\pause
\item \textbf{TEE} (Trans Europa Express): \\
    {\em Hier kommt noch Text rein} 
\end{itemize}
\end{frame}


\begin{frame}
\frametitle{Dritte Folie im ersten Abschnitt}
Diese Folie ist mit der columns--Umgebung mehrspaltig gesetzt. Die Spaltenbreite kann frei gew�hlt werden (hier: linke Spalte \texttt{0.3}$\backslash$\texttt{textwidth}, rechte Spalte \texttt{0.6}$\backslash$\texttt{textwidth}). Die Schriftgr��e in den zwei Listen ist ausnahmsweise $\backslash$\texttt{small}.

\pause
\begin{columns}\small\em 
% Umschaltung auf small und emphasize innerhalb der columns-Umgebung 
\column{.3\textwidth}
\begin{itemize}
\item Booklettext
\item �bersetzung
\item K�nstlerfotos 
\item Trackliste 
\item Labelpass
\item Textredaktion
\end{itemize}
\column{.6\textwidth}
\begin{itemize}
\item Coverdesign
\item Booklet--Layout
\item CD--Label--Design
\item Digitalproof
\item GEMA--Anmeldung
\item \bfseries{und ab ans Presswerk!}
\end{itemize}
\end{columns}
\end{frame}



\begin{frame}
\frametitle{Die letzte Folie im ersten Abschnitt}
...mit Bildern im zweispaltigen Satz. Die Einbindung der Bilder wird hier mit dem Befehl pgfimage realisiert. Die Ma�angabe $\backslash$\texttt{textwidth} ist relativ: au�erhalb des zweispaltigen Satzes bezieht sie sich auf die gesamte zur Verf�gung stehende Breite, innerhalb auf die aktuelle Spaltenbreite.

\medskip %mittelgro�er vertikaler Abstand zwischen Text und Bildern

\begin{columns}

\column{.4\textwidth}
\begin{center}
\pgfimage[width=0.9\textwidth]{Barcode} \\
Ein skaliertes JPG--Bild 
\end{center}

\column{.4\textwidth}
\begin{center}
\pgfimage[width=0.6\textwidth]{CD_Audio_logo}  \\
Ein skaliertes PDF--Bild
\end{center}

\end{columns}

\end{frame}


%%%%%%%%%%%%%%%%%%%%%%%%%%%%%%%%%%%%%%%%%%%
\section{Zweiter Abschnitt}
%%%%%%%%%%%%%%%%%%%%%%%%%%%%%%%%%%%%%%%%%%%
\begin{frame}
\frametitle{Erste Folie im zweiten Abschnitt}

\pause
\begin{block}{Diese Folie}
wird aus Standard-Bl�cken (Umgebung block) zusammengesetzt:
\end{block}

\pause
\begin{block}{Noch ein Block}
Hier kommt noch Text rein
\end{block}

\pause
\begin{block}{Und der letzte Block} 
Hier kommt noch Text rein
\end{block}
\end{frame}


\begin{frame}
\frametitle{Zweite Folie im zweiten Abschnitt}

...mit einem exampleblock (gr�n) und einem alertblock (rot):

\pause
\begin{exampleblock}{Beispiel: exampleblock}
...Neuigkeiten zum Merken\\
...mehr Neuigkeiten zum Merken
\end{exampleblock}

\pause
\begin{alertblock}{Beispiel: alertblock} 
wichtige Dinge auf die mal hingewiesen werden muss
\end{alertblock}

\pause
Mehr als drei Farben (Standardfarbe --- hier Blau --- plus Gr�n plus Rot f�r besonders wichtige Punkte) sollten nicht benutzt werde, sonst wirds zu bunt.

\end{frame}


%%%%%%%%%%%%%%%%%%%%%%%%%%%%%%%%%%%%%%%%%%%
\section{Dritter Abschnitt}
%%%%%%%%%%%%%%%%%%%%%%%%%%%%%%%%%%%%%%%%%%%
\begin{frame}
\frametitle{Erste Folie im dritten Abschnitt}
Ein bisschen Mathematik, und dabei der pause-Befehl:
\pause
\begin{eqnarray}\nonumber
\xi_1(t) + \xi_2(t) &=& \xi_0\,\left(\sin(\omega_1 t)+\sin(\omega_2 t)\right) \\ \nonumber
    &=& 2\xi_0\, \underbrace{\sin\left(\frac{(\omega_1+\omega_2)\, t}{2}\right)}_{\mbox{\scriptsize Schwingung mit $\frac{\omega_1+\omega_2}{2}$}}\cdot \underbrace{\cos\left(\frac{(\omega_1-\omega_2)\, t}{2}\right)}_{\mbox{\scriptsize Modulation mit $\frac{\omega_1-\omega_2}{2}$}}
\end{eqnarray}
Text, der hinter der Gleichung steht.
\end{frame}



\begin{frame}
\frametitle{Zweite Folie im dritten Abschnitt}
Innerhalb der Formel ist der uncover-Befehl eleganter:
\begin{eqnarray}\nonumber
\uncover<2->{\xi_1(t) + \xi_2(t)} &\uncover<2->{=}& \uncover<2->{\xi_0\,\left(\sin(\omega_1 t)+\sin(\omega_2
t)\right)} \\ \nonumber
    &\uncover<3->{=}& \uncover<3->{2\xi_0\, \underbrace{\sin\left(\frac{(\omega_1+\omega_2)\, t}{2}\right)}_{\mbox{\scriptsize\uncover<4->{Schwingung mit $\frac{\omega_1+\omega_2}{2}$}}}\cdot
    \underbrace{\cos\left(\frac{(\omega_1-\omega_2)\, t}{2}\right)}_{\mbox{\scriptsize\uncover<5->{Modulation mit $\frac{\omega_1-\omega_2}{2}$}}}}
\end{eqnarray}

$\backslash$\texttt{uncover}$<$nr-$>$$\{$...$\}$ bekommt als Parameter die Nummer der Teil--Folie �bergeben, ab der der geklammerte Bereich sichtbar wird. $\backslash$\texttt{uncover}$<$nr$>$$\{$...$\}$ enth�llt den Text nur auf einer Teilfolie, $\backslash$\texttt{uncover}$<$nr1-nr2$>$$\{$...$\}$ von der Teil--Folie nr1 bis zur Teil--Folie nr2.
\end{frame}



\begin{frame}
\frametitle{Dritte Folie im dritten Abschnitt}
Der only-Befehl funktioniert �hnlich wie uncover:
\begin{eqnarray}\nonumber
\only<2->{\xi_1(t) + \xi_2(t)} &\only<2->{=}& \only<2->{\xi_0\,\left(\sin(\omega_1 t)+\sin(\omega_2
t)\right)} \\ \nonumber
    &\only<3->{=}& \only<3->{2\xi_0\, \underbrace{\sin\left(\frac{(\omega_1+\omega_2)\, t}{2}\right)}_{\mbox{\scriptsize\only<4->{Schwingung mit $\frac{\omega_1+\omega_2}{2}$}}}\cdot
    \underbrace{\cos\left(\frac{(\omega_1-\omega_2)\, t}{2}\right)}_{\mbox{\scriptsize\only<5>{Modulation mit $\frac{\omega_1-\omega_2}{2}$}}}}
\end{eqnarray}

Allerdings l�sst $\backslash$\texttt{only}$<$nr-$>$$\{$...$\}$ den verborgenen Text komplett verschwinden; damit wird auch kein Platz reserviert. 
\end{frame}



\begin{frame}
\frametitle{Vierte Folie im dritten Abschnitt}
Und hier die Wirkung des (komplement�ren) invisible--Befehls:
\begin{eqnarray}\nonumber
\only<2->{\xi_1(t) + \xi_2(t)} &\only<2->{=}& \only<2->{\xi_0\,\left(\sin(\omega_1 t)+\sin(\omega_2
t)\right)} \\ \nonumber
    &\invisible<1-2>{=}& \invisible<1-2>{2\xi_0\, \underbrace{\sin\left(\frac{(\omega_1+\omega_2)\, t}{2}\right)}_{\mbox{\scriptsize\alert<4>{Schwingung mit $\frac{\omega_1+\omega_2}{2}$}}}\cdot
    \underbrace{\cos\left(\frac{(\omega_1-\omega_2)\, t}{2}\right)}_{\mbox{\scriptsize\alert<5>{Modulation mit $\frac{\omega_1-\omega_2}{2}$}}}}
\end{eqnarray}

Mit $\backslash$\texttt{invisible}$<$nr1-nr2$>$$\{$...$\}$ wird der geklammerte Text von Teil--Folie nr1 bis nr2 ebenfalls ganz unsichtbar; der Platz auf der Folie bleibt aber reserviert. Und mit $\backslash$\texttt{alert}$<$nr$>$$\{$...$\}$ kann man Textteile auf einer Teil--Folie rot markieren. 
\end{frame}



%%%%%%%%%%%%%%%%%%%%%%%%%%%%%%%%%%%%%%%%%%%
\section*{Zusammen\-fassung}
%%%%%%%%%%%%%%%%%%%%%%%%%%%%%%%%%%%%%%%%%%%
\begin{frame}
\frametitle{Zusammenfassung}
Vor diesem letzten Abschnitt kommt kommt {\em kein} automatisch eingeschobener Sprung ins Inhaltsverzeichnis, weil hier die gesternte Form $\backslash$\texttt{section*}$\{$...$\}$ benutzt wird.
\end{frame}


\begin{frame}
\frametitle{Zusammenfassung}
Und zum Schluss noch die

\begin{center}
\underbar{Goldene Folien--Regel}
\end{center}

\invisible<1>{
\begin{alertblock}{Goldene Regel}
Wenn der geplante Inhalt im Standard--Layout nicht auf eine Folie passt, dann sollte man \alert<3>{nicht am Layout herumwurschteln} und schon gar nicht die Schrift verkleinern, \alert<3>{sondern den Inhalt der Folie k�rzen}. Ganze S�tze geh�ren nicht auf die Folie.
\end{alertblock}}

\uncover<4->{
Deshalb:
\begin{itemize}
\alert<5>{\item besser Aufz�hlungen als Flie�text!}
\alert<6>{\item nur Stichworte!}
\alert<7>{\item niemals ganze S�tze!}
\end{itemize}}

\end{frame}


\begin{frame}
\frametitle{}

\centering
Herzlichen Dank f�r Ihre Aufmerksamkeit!

\end{frame}


\end{document}



%%%%%%%%%%%%%%%%%%%%%%%%%%%%%%%%%%%%%%%%%%%
%%  besondere Gestaltungselemente f�r Vortragsfolien
%%%%%%%%%%%%%%%%%%%%%%%%%%%%%%%%%%%%%%%%%%%

% Folie
\begin{frame}
\frametitle{Folientitel}
        Text, Bilder, Bl�cke
\end{frame}

% neutraler Kasten
\begin{block}{Blocktitel}
        Blocktext
\end{block}

% gr�ner Kasten
\begin{exampleblock}{Beispielblocktitel}
        Beispielblocktext
\end{exampleblock}

% roter Kasten
\begin{alertblock}{Warnungsblocktitel}
        Warnungsblocktext
\end{alertblock}

% mehrspaltige Folie mit variablen Spaltenbreiten
\begin{columns}
\column{.55\textwidth}
        Text oder Bild
\column{.45\textwidth}
        Text oder Bild
\end{columns}

% sukzessiver Folienaufbau 
\pause

% mehr Befehle f�r sukzessiven Folienaufbau, 
% der geklammerte Teil erscheint jeweils ab der dritten Teil-Folie
\uncover<3->{}
\only<3->{}
\invisible<1-2>{}

% rote Markierung des geklammerten Teils bei der vierten Teil-Folie
\alert<4>{}

%%%%%%%%%%%%%%%%%%%%%%%%%%%%%%%%%%%%%%%%%%%